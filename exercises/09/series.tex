\documentclass[a4paper]{scrreprt}

% Uncomment to optimize for double-sided printing.
% \KOMAoptions{twoside}

% Set binding correction manually, if known.
% \KOMAoptions{BCOR=2cm}

% Localization options
\usepackage[english]{babel}
\usepackage[T1]{fontenc}
\usepackage[utf8]{inputenc}

% Quotations
\usepackage{dirtytalk}

% Floats
\usepackage{float}

% Enhanced verbatim sections. We're mainly interested in
% \verbatiminput though.
\usepackage{verbatim}

% Automatically remove leading whitespace in lstlisting
\usepackage{lstautogobble}

% PDF-compatible landscape mode.
% Makes PDF viewers show the page rotated by 90°.
\usepackage{pdflscape}

% Advanced tables
\usepackage{array}
\usepackage{tabularx}
\usepackage{longtable}

% Fancy tablerules
\usepackage{booktabs}

% Graphics
\usepackage{graphicx}

% Current time
\usepackage[useregional=numeric]{datetime2}

% Float barriers.
% Automatically add a FloatBarrier to each \section
\usepackage[section]{placeins}

% Custom header and footer
\usepackage{fancyhdr}

\usepackage{geometry}
\usepackage{layout}

% Math tools
\usepackage{mathtools}
% Math symbols
\usepackage{amsmath,amsfonts,amssymb}
\usepackage{amsthm}
% General symbols
\usepackage{stmaryrd}

\DeclarePairedDelimiter\abs{\lvert}{\rvert}

% Indistinguishable operator (three stacked tildes)
\newcommand*{\diffeo}{% 
  \mathrel{\vcenter{\offinterlineskip
  \hbox{$\sim$}\vskip-.35ex\hbox{$\sim$}\vskip-.35ex\hbox{$\sim$}}}}

% Bullet point
\newcommand{\tabitem}{~~\llap{\textbullet}~~}

\pagestyle{plain}
% \fancyhf{}
% \lhead{}
% \lfoot{}
% \rfoot{}
% 
% Source code & highlighting
\usepackage{listings}

% SI units
\usepackage[binary-units=true]{siunitx}
\DeclareSIUnit\cycles{cycles}

% Convenience commands
\newcommand{\mailsubject}{11043 - Internet of Things - Series 9}
\newcommand{\maillink}[1]{\href{mailto:#1?subject=\mailsubject}
                               {#1}}

% Should use this command wherever the print date is mentioned.
\newcommand{\printdate}{\today}

\subject{11043 - Internet of Things}
\title{Series 9}

\author{Michael Senn \maillink{michael.senn@students.unibe.ch} - 16-126-880}

\date{\printdate}

% Needs to be the last command in the preamble, for one reason or
% another. 
\usepackage{hyperref}

\begin{document}
\maketitle


\setcounter{chapter}{8}

\chapter{Series 9}

\section{Hop-by-hop vs end-to-end with RMST}

As each hop has a certain error transmission percentage, the overall
(end-to-end) error transmission percentage increases significantly as the
number of hops increases. This applies not only to the initial packet, but also
to any subsequen transmission. As an example consider and end-to-end arrival
probability of \SI{40}{\percent}. Not only is the initial packet likely to not
arrive, but any kind of end-to-end retransmission is also likely to not arrive.
This greatly increase the total amount of packets which need to be sent.

With a hop-by-hop approach, the amount of total transmissions is significantly
lower as a retransmission can happen immediately between the two hops where
transmission failed. This way, a failed transmission somewhere along the route
does not require a full retransmission between source and sink.

\section{TCP in IoT}

TCP was designed for connections with a low error rate, which does not
generally hold for IoT networks. In IoT networks, the combination of wireless
communication (susceptible to environmental conditions and medium overload),
along with an unstable topology and unreliable nodes, means that error rates
are often high.

Specifically this leads to a few issues, such as:
\begin{itemize}
		\item TCP does not distinguish losses due to congestion from losses due
				to transmission errors. In IoT networks it is desirable to be
				able to handle each case separately, via specialized protocols
				and mechanisms, as their root cause is different.
		\item TCP performs congestion control and error recovery only in the
				endpoints of a connection, which --- in networks with high
				error rates --- causes a large number of transmissions to
				occurr. In IoT networks it is desireable to be able to --- at
				least selectively --- have a hop-by-hop handling of such
				issues.
		\item TCP by construction always offers reliable transport, which
				causes a certain overhead. In IoT networks there are cases
				where unreliable transport mechanisms are acceptable, in order
				to lower resource usage.
\end{itemize}

\end{document}
