\documentclass[a4paper]{scrreprt}

% Uncomment to optimize for double-sided printing.
% \KOMAoptions{twoside}

% Set binding correction manually, if known.
% \KOMAoptions{BCOR=2cm}

% Localization options
\usepackage[english]{babel}
\usepackage[T1]{fontenc}
\usepackage[utf8]{inputenc}

% Quotations
\usepackage{dirtytalk}

% Floats
\usepackage{float}

% Enhanced verbatim sections. We're mainly interested in
% \verbatiminput though.
\usepackage{verbatim}

% Automatically remove leading whitespace in lstlisting
\usepackage{lstautogobble}

% PDF-compatible landscape mode.
% Makes PDF viewers show the page rotated by 90°.
\usepackage{pdflscape}

% Advanced tables
\usepackage{array}
\usepackage{tabularx}
\usepackage{longtable}

% Fancy tablerules
\usepackage{booktabs}

% Graphics
\usepackage{graphicx}

% Current time
\usepackage[useregional=numeric]{datetime2}

% Float barriers.
% Automatically add a FloatBarrier to each \section
\usepackage[section]{placeins}

% Custom header and footer
\usepackage{fancyhdr}

\usepackage{geometry}
\usepackage{layout}

% Math tools
\usepackage{mathtools}
% Math symbols
\usepackage{amsmath,amsfonts,amssymb}
\usepackage{amsthm}
% General symbols
\usepackage{stmaryrd}

\DeclarePairedDelimiter\abs{\lvert}{\rvert}

% Indistinguishable operator (three stacked tildes)
\newcommand*{\diffeo}{% 
  \mathrel{\vcenter{\offinterlineskip
  \hbox{$\sim$}\vskip-.35ex\hbox{$\sim$}\vskip-.35ex\hbox{$\sim$}}}}

% Bullet point
\newcommand{\tabitem}{~~\llap{\textbullet}~~}

\floatstyle{ruled}
\newfloat{algo}{htbp}{algo}
\floatname{algo}{Algorithm}
% For use in algorithms
\newcommand{\str}[1]{\textsc{#1}}
\newcommand{\var}[1]{\textit{#1}}
\newcommand{\op}[1]{\textsl{#1}}

\pagestyle{plain}
% \fancyhf{}
% \lhead{}
% \lfoot{}
% \rfoot{}
% 
% Source code & highlighting
\usepackage{listings}

% SI units
\usepackage[binary-units=true]{siunitx}
\DeclareSIUnit\cycles{cycles}

% Convenience commands
\newcommand{\mailsubject}{11043 - Internet of Things - Series 3}
\newcommand{\maillink}[1]{\href{mailto:#1?subject=\mailsubject}
                               {#1}}

% Should use this command wherever the print date is mentioned.
\newcommand{\printdate}{\today}

\subject{11043 - Internet of Things}
\title{Series 3}

\author{Michael Senn \maillink{michael.senn@students.unibe.ch} - 16-126-880}

\date{\printdate}

% Needs to be the last command in the preamble, for one reason or
% another. 
\usepackage{hyperref}

\begin{document}
\maketitle


\setcounter{chapter}{2}

\chapter{Series 3}

\section{Multi-threaded versus event-driven concurrency}

\begin{itemize}
		\item With event-driven concurrency, an invoked event-handler is
				executed until completion. Only when it finished is another
				invoked. With multi-threaded concurrency however, threads will
				run until the next blocking statement, at which point another
				will take over.
		\item With event-driven concurrency, a single stack is sufficient as
				each event is run until execution, so no state must be
				persisted. With multi-threaded concurrency each thread needs
				its own stack, in order to persist state as its interruption
				will be interrupted multiple times.
\end{itemize}

\section{LEDF scheduling}

Assume that the scheduling algorithm runs whenever a new task is scheduled, or
a scheduled task finished. Assume that the deadlines are relative to the
thread's arrival times.

Given the processor speeds in table \ref{tbl:processor_speeds}, tasks in table
\ref{tbl:tasks} and algorithm from slide 39 of the lecture, the execution plan
in table \ref{tbl:execution_plan} results. Entries of the task queue are tuples
$(t_i, l_i)$, with $l_i$ the \textbf{remaining} instruction length of thread
$t_i$ at the given time. The task queue is ordered by ascending deadline.

\begin{table}
		\centering
		\begin{tabular}{lll}
				\toprule
				Time & Choice of $S$ & Task queue \\
				\midrule
				\SI{0}{\s} & 300 & $((t_1, 900))$ \\
				\SI{2}{\s} & 200 & $((t_1, 300), (t_2, 1800))$ \\
				\SI{2.5}{\s} & 200 & $((t_1, 200), (t_2, 1800), (t_3, 450))$ \\
				\SI{3.5}{\s} & 300 & $((t_2, 1800), (t_3, 450))$ \\
				\SI{5}{\s} & 300 & $((t_2, 1350), (t_3, 450), (t_4, 1000))$ \\
				\SI{6}{\s} & 300 & $((t_2, 1050), (t_3, 450), (t_5, 800), (t_4, 1000))$ \\
				\SI{9.5}{\s} & 300 & $((t_3, 450), (t_5, 800), (t_4, 1000))$ \\
				\SI{10.5}{\s} & 200 & $((t_5, 800), (t_4, 1000))$ \\
				\SI{14.5}{\s} & 200 & $((t_4, 1000))$ \\
				\SI{19.5}{\s} & - & $()$ \\
				\bottomrule
		\end{tabular}
		\caption{Execution plan}
		\label{tbl:execution_plan}
\end{table}

\begin{table}
		\centering
		\begin{tabular}{ll}
				\toprule
				Processor speed [MIPS] & Voltage [V] \\
				\midrule
				200 & 1.5 \\
				300 & 2 \\
				450 & 3.5 \\
				\bottomrule
		\end{tabular}
		\caption{Processor speeds}
		\label{tbl:processor_speeds}
\end{table}

\begin{table}
		\centering
		\begin{tabular}{lllll}
				\toprule
				Task & Arrival time [s] & Deadline (rel) [s] & Deadline (abs) [s] & Length [MI] \\
				\midrule
				$t_1$ & 0 & 4 & 4 & 900 \\
				$t_2$ & 2 & 8 & 10 & 1800 \\
				$t_3$ & 2.5 & 8 & 10.5 & 450 \\
				$t_4$ & 5 & 20 & 25 & 1000 \\
				$t_5$ & 6 & 14 & 20 & 800 \\
				\bottomrule
		\end{tabular}
		\caption{Task arrival}
		\label{tbl:tasks}
\end{table}

\end{document}
