\documentclass[a4paper]{scrreprt}

% Uncomment to optimize for double-sided printing.
% \KOMAoptions{twoside}

% Set binding correction manually, if known.
% \KOMAoptions{BCOR=2cm}

% Localization options
\usepackage[english]{babel}
\usepackage[T1]{fontenc}
\usepackage[utf8]{inputenc}

% Quotations
\usepackage{dirtytalk}

% Floats
\usepackage{float}

% Enhanced verbatim sections. We're mainly interested in
% \verbatiminput though.
\usepackage{verbatim}

% Automatically remove leading whitespace in lstlisting
\usepackage{lstautogobble}

% PDF-compatible landscape mode.
% Makes PDF viewers show the page rotated by 90°.
\usepackage{pdflscape}

% Advanced tables
\usepackage{array}
\usepackage{tabularx}
\usepackage{longtable}

% Fancy tablerules
\usepackage{booktabs}

% Graphics
\usepackage{graphicx}

% Current time
\usepackage[useregional=numeric]{datetime2}

% Float barriers.
% Automatically add a FloatBarrier to each \section
\usepackage[section]{placeins}

% Custom header and footer
\usepackage{fancyhdr}

\usepackage{geometry}
\usepackage{layout}

% Math tools
\usepackage{mathtools}
% Math symbols
\usepackage{amsmath,amsfonts,amssymb}
\usepackage{amsthm}
% General symbols
\usepackage{stmaryrd}

\DeclarePairedDelimiter\abs{\lvert}{\rvert}

% Indistinguishable operator (three stacked tildes)
\newcommand*{\diffeo}{% 
  \mathrel{\vcenter{\offinterlineskip
  \hbox{$\sim$}\vskip-.35ex\hbox{$\sim$}\vskip-.35ex\hbox{$\sim$}}}}

% Bullet point
\newcommand{\tabitem}{~~\llap{\textbullet}~~}

\floatstyle{ruled}
\newfloat{algo}{htbp}{algo}
\floatname{algo}{Algorithm}
% For use in algorithms
\newcommand{\str}[1]{\textsc{#1}}
\newcommand{\var}[1]{\textit{#1}}
\newcommand{\op}[1]{\textsl{#1}}

\pagestyle{plain}
% \fancyhf{}
% \lhead{}
% \lfoot{}
% \rfoot{}
% 
% Source code & highlighting
\usepackage{listings}

% SI units
\usepackage[binary-units=true]{siunitx}
\DeclareSIUnit\cycles{cycles}

% Convenience commands
\newcommand{\mailsubject}{11043 - Internet of Things - Series 2}
\newcommand{\maillink}[1]{\href{mailto:#1?subject=\mailsubject}
                               {#1}}

% Should use this command wherever the print date is mentioned.
\newcommand{\printdate}{\today}

\subject{11043 - Internet of Things}
\title{Series 2}

\author{Michael Senn \maillink{michael.senn@students.unibe.ch} - 16-126-880}

\date{\printdate}

% Needs to be the last command in the preamble, for one reason or
% another. 
\usepackage{hyperref}

\begin{document}
\maketitle


\setcounter{chapter}{1}

\chapter{Series 2}

\section{Multi-hop communication}

Multi-hop communication is a network communication model where, instead of data
being exchanged directly between two nodes $A$ and $B$, it is exchanged via a
set of intermediary nodes, such that any individual hop covers a shorter
distance than direct communication would.

As energy usage is cubic - or worse - with distance covered, multi-hop
communication becomes more advantageous the longer the distance to cover. As
each additional hop also incurs a certain energy overhead, too many hops is not
beneficial however.

\section{Saving energy by sleeping}

Recall the two equations from the lecture, for the energy overhead incurred and
the energy saved by sleeping:

\begin{align*}
		E_{saved} & = (t_{event} - t_1) \cdot P_{active} - (t_{down} \cdot (P_{active} + P_{sleep}) / 2 + (t_{event} - t_1 - t_{down}) \cdot P_{sleep}) \\
		E_{overhead} & = t_{up} \cdot(P_{active} + P_{sleep}) / 2
\end{align*}

Then the minimum sleep duration $t_{sleep} := t_{event} - t_{1}$ for which a
sleep saves at least as much energy as it costs can be found by solving
$E_{saved} = E_{overhead}$ for $t_{sleep}$. Any sleep longer than $t_{sleep}$
will then save more energy than the overhead.

Assume that the voltage in both states is constant, in which case the result
will be independent of the voltage. To simplify calculations we assume the
voltage to be \SI{1}{\V}.

\subsection{First example}

Let $t_{down} = \SI{3}{\ms}$, $t_{up} = \SI{3}{\ms}$, $I_{active} = \SI{15}{\mA}$, $I_{sleep} = \SI{25}{\uA}$

Then:
\begin{align*}
		E_{overhead} & = \SI{2.25375e-5}{\J} = E_{saved} \\
		\Rightarrow t_{sleep} & \approx \SI{3.005}{\ms}
\end{align*}

That is, any sleep duration longer than \SI{3.005}{\ms} will lead to a net
save of energy.

\subsection{Second example}

Let $t_{down} = \SI{4.5}{\ms}$, $t_{up} = \SI{4.5}{\ms}$, $I_{active} = \SI{10}{\mA}$, $I_{sleep} = \SI{15}{\uA}$

Then:
\begin{align*}
		E_{overhead} & = \SI{2.25338e-5}{\J} = E_{saved} \\
		\Rightarrow t_{sleep} & \approx \SI{4.5068}{\ms}
\end{align*}

That is, any sleep duration longer than \SI{4.5068}{\ms} will lead to a net
save of energy.

\section{Power usage \& battery capacity}

Assume that, when transmitting data, the microcontroller must be in the active
state.

\subsection{Average power consumption}

First consider the time spent transmitting data. With a
\SI{19.2}{\kilo\bit\per\second} bandwith, transmitting a \SI{96}{\byte} packet
takes \SI{40}{\ms}.

As such, per \SI{1}{\s} the node will spend:

\begin{itemize}
		\item \SI{10}{\ms} measuring and processing data, with a current of \SI{20}{\mA}
		\item \SI{40}{\ms} transmitting data, with a current of \SI{30}{\mA}
		\item \SI{950}{\ms} idle, with a current of \SI{250}{\uA}
\end{itemize}

As such its average power consumption will be $\SI{5}{\V} \cdot \SI{1.638}{\mA}
= \SI{8.19}{\mW}$.

\subsection{Battery power}

Three \SI{5}{\V} batteries of \SI{1000}{\milli\ampere\hour} will, in our
simplified model, provide \SI{3000}{\milli\ampere\hour} at \SI{5}{\V}. At a
consumption of \SI{1.638}{\milli\ampere\hour} at \SI{5}{\V}, this will last for
approximately \SI{76.31}{\day}.

\end{document}
