\documentclass[a4paper]{scrreprt}

% Uncomment to optimize for double-sided printing.
% \KOMAoptions{twoside}

% Set binding correction manually, if known.
% \KOMAoptions{BCOR=2cm}

% Localization options
\usepackage[english]{babel}
\usepackage[T1]{fontenc}
\usepackage[utf8]{inputenc}

% Quotations
\usepackage{dirtytalk}

% Floats
\usepackage{float}

% Enhanced verbatim sections. We're mainly interested in
% \verbatiminput though.
\usepackage{verbatim}

% Automatically remove leading whitespace in lstlisting
\usepackage{lstautogobble}

% PDF-compatible landscape mode.
% Makes PDF viewers show the page rotated by 90°.
\usepackage{pdflscape}

% Advanced tables
\usepackage{array}
\usepackage{tabularx}
\usepackage{longtable}

% Fancy tablerules
\usepackage{booktabs}

% Graphics
\usepackage{graphicx}

% Current time
\usepackage[useregional=numeric]{datetime2}

% Float barriers.
% Automatically add a FloatBarrier to each \section
\usepackage[section]{placeins}

% Custom header and footer
\usepackage{fancyhdr}

\usepackage{geometry}
\usepackage{layout}

% Math tools
\usepackage{mathtools}
% Math symbols
\usepackage{amsmath,amsfonts,amssymb}
\usepackage{amsthm}
% General symbols
\usepackage{stmaryrd}

\DeclarePairedDelimiter\abs{\lvert}{\rvert}

% Indistinguishable operator (three stacked tildes)
\newcommand*{\diffeo}{% 
  \mathrel{\vcenter{\offinterlineskip
  \hbox{$\sim$}\vskip-.35ex\hbox{$\sim$}\vskip-.35ex\hbox{$\sim$}}}}

% Bullet point
\newcommand{\tabitem}{~~\llap{\textbullet}~~}

\floatstyle{ruled}
\newfloat{algo}{htbp}{algo}
\floatname{algo}{Algorithm}
% For use in algorithms
\newcommand{\str}[1]{\textsc{#1}}
\newcommand{\var}[1]{\textit{#1}}
\newcommand{\op}[1]{\textsl{#1}}

\pagestyle{plain}
% \fancyhf{}
% \lhead{}
% \lfoot{}
% \rfoot{}
% 
% Source code & highlighting
\usepackage{listings}

% SI units
\usepackage[binary-units=true]{siunitx}
\DeclareSIUnit\cycles{cycles}

% Convenience commands
\newcommand{\mailsubject}{11043 - Internet of Things - Series 5}
\newcommand{\maillink}[1]{\href{mailto:#1?subject=\mailsubject}
                               {#1}}

% Should use this command wherever the print date is mentioned.
\newcommand{\printdate}{\today}

\subject{11043 - Internet of Things}
\title{Series 5}

\author{Michael Senn \maillink{michael.senn@students.unibe.ch} - 16-126-880}

\date{\printdate}

% Needs to be the last command in the preamble, for one reason or
% another. 
\usepackage{hyperref}

\begin{document}
\maketitle


\setcounter{chapter}{4}

\chapter{Series 5}

\section{Requirements for time synchronization}

Due to the limitations of sensor nodes, specific requirements exist when it
comes to time synchronization.

\begin{description}
		\item[Energy efficiency] As nodes might not have access to the power
				grid, energy usage must be low to be feasible to allow eg
				operation with batteries.
		\item[Cost \& size] As nodes might be deployed en masse, involved
				components must be cheap enough to keep the price of each
				individual node low. Similarly nodes might operate in
				space-constrained environments, so e.g. big antennas might not
				be feasible.
		\item[Robustness] Sensor nodes may be added and removed randomly, and
				existing ones might die. As such, time synchronization schemes
				must be able to handle such a dynamic environment without
				breaking down.
\end{description}

\section{Time drift \& skew example}

Let $A, B$ be two nodes which booted at the same time. At a later time $t$, A's
time is $C_A(t) = \SI{100}{\s}$, B's time is $C_B(t) = \SI{105}{\s}$.

\subsection{Clock drift of B from A}

The clock drift is defined as the `difference between a clock and a reference
per unit of time of the reference'. Assuming $A$ to be the reference, $B$ has
drifted \SI{5}{\s} over a total of \SI{100}{\s} of $A$, leading to a drift of:

\[
		d_B = \frac{\SI{5}{\s}}{\SI{100}{\s}} = \SI{5e-2}{\s \per \s}
\]

\subsection{Linear relation between two clocks}

As the two nodes were started simultaneously, their initial offsets are $0$.
With $d_i = \rho_i + 1$, where $\rho_i$ is the drift of node $i$, this
simplifies the linear relation between the two to:

\begin{align*}
		C_A(t) & = d_{A, B} \cdot C_B(t) = \frac{d_A}{d_B} \cdot C_B(t) \approx \SI{0.952}{\s \per \s} \cdot C_B(t) \\
		C_B(t) & = d_{B, A} \cdot C_A(t) = \frac{d_B}{d_A} \cdot C_A(t) = \SI{1.05}{\s \per \s} \cdot C_A(t) \\
\end{align*}

\subsection{Clock skew of B to A}

The clock skew is the difference of two clocks. As such it is given as:

\[
		s_B = C_B(t) - C_A(t) = \SI{105}{\s} - \SI{100}{\s} = \SI{5}{\s}
\]

\end{document}
