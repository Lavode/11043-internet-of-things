\documentclass[a4paper]{scrreprt}

% Uncomment to optimize for double-sided printing.
% \KOMAoptions{twoside}

% Set binding correction manually, if known.
% \KOMAoptions{BCOR=2cm}

% Localization options
\usepackage[english]{babel}
\usepackage[T1]{fontenc}
\usepackage[utf8]{inputenc}

% Quotations
\usepackage{dirtytalk}

% Floats
\usepackage{float}

% Enhanced verbatim sections. We're mainly interested in
% \verbatiminput though.
\usepackage{verbatim}

% Automatically remove leading whitespace in lstlisting
\usepackage{lstautogobble}

% PDF-compatible landscape mode.
% Makes PDF viewers show the page rotated by 90°.
\usepackage{pdflscape}

% Advanced tables
\usepackage{array}
\usepackage{tabularx}
\usepackage{longtable}

% Fancy tablerules
\usepackage{booktabs}

% Graphics
\usepackage{graphicx}

% Current time
\usepackage[useregional=numeric]{datetime2}

% Float barriers.
% Automatically add a FloatBarrier to each \section
\usepackage[section]{placeins}

% Custom header and footer
\usepackage{fancyhdr}

\usepackage{geometry}
\usepackage{layout}

% Math tools
\usepackage{mathtools}
% Math symbols
\usepackage{amsmath,amsfonts,amssymb}
\usepackage{amsthm}
% General symbols
\usepackage{stmaryrd}

\DeclarePairedDelimiter\abs{\lvert}{\rvert}

% Indistinguishable operator (three stacked tildes)
\newcommand*{\diffeo}{% 
  \mathrel{\vcenter{\offinterlineskip
  \hbox{$\sim$}\vskip-.35ex\hbox{$\sim$}\vskip-.35ex\hbox{$\sim$}}}}

% Bullet point
\newcommand{\tabitem}{~~\llap{\textbullet}~~}

\pagestyle{plain}
% \fancyhf{}
% \lhead{}
% \lfoot{}
% \rfoot{}
% 
% Source code & highlighting
\usepackage{listings}

% SI units
\usepackage[binary-units=true]{siunitx}
\DeclareSIUnit\cycles{cycles}

% Convenience commands
\newcommand{\mailsubject}{11043 - Internet of Things - Series 11}
\newcommand{\maillink}[1]{\href{mailto:#1?subject=\mailsubject}
                               {#1}}

% Should use this command wherever the print date is mentioned.
\newcommand{\printdate}{\today}

\subject{11043 - Internet of Things}
\title{Series 11}

\author{Michael Senn \maillink{michael.senn@students.unibe.ch} - 16-126-880}

\date{\printdate}

% Needs to be the last command in the preamble, for one reason or
% another. 
\usepackage{hyperref}

\begin{document}
\maketitle


\setcounter{chapter}{10}

\chapter{Series 11}

\section{Symmetric vs asymmetric cryptography in IoT networks}

With asymmetric cryptography, each node posesses two keys. A public key, which
is known to everyone, and a private key which is kept to itself. To ensure
authenticity, all public keys have to either be known to every node in advance,
or they have to e.g. be signed by a shared trusted entity.

This approach has the benefit of being fully scalable. When a node node is
deployed, it simply receives a new keypair, and will be able to communicate
securely with other nodes in the network. Its downsides are, for one, the
comparably large computational cost of asymmetric cryptography, which is
especially relevant for computationally-weak environments as found in IoT. Its
second downside is that, if a node was to be captured, an attacker could then
impersonate this node. Protocols for key revocation exist, but do require nodes
to e.g. periodically query a central entity for blacklisted keys.

In the case of symmetric cryptography, each pair of nodes has a shared key
which they use for communication between the two. This has the advantage of
allowing the use of fast symmetric encryption protocols, which benefit IoT
devices.

Its main disadvantage is the inherent lack of scalability. If a node is added,
all nodes must be able to agree on a shared key with the new node. This
requires either updates to all existing nodes, or the use of a central
authority which distributes these keys.

A realistic approach might thus use symmetric keys for the bulk of the
communication, with asymmetric cryptography used to agree on the keys to use
for each pair.

\section{Main layers of cyber-physical systems}

The lowest layer of a CPS is the perception layer. It contains sensors and
actuators, i.e. any components which actually interface with the physical world.
An example of a component in this layer would be a temperature sensors, or a
valve actuator.

The middle layer is the transmission layer, which is concerned with allowing
components of the perception layer to communicate with each other, as well as
with any application. Examples of this layer are network hardware such as
routers, or protocols such as WiFi.

The upmost layer is the application layer. In it there are applications which
collect data from sensors, perform calculations, and send commands to
actuators.  An example would be a city's smart energy grid.

\end{document}
