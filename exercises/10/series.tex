\documentclass[a4paper]{scrreprt}

% Uncomment to optimize for double-sided printing.
% \KOMAoptions{twoside}

% Set binding correction manually, if known.
% \KOMAoptions{BCOR=2cm}

% Localization options
\usepackage[english]{babel}
\usepackage[T1]{fontenc}
\usepackage[utf8]{inputenc}

% Quotations
\usepackage{dirtytalk}

% Floats
\usepackage{float}

% Enhanced verbatim sections. We're mainly interested in
% \verbatiminput though.
\usepackage{verbatim}

% Automatically remove leading whitespace in lstlisting
\usepackage{lstautogobble}

% PDF-compatible landscape mode.
% Makes PDF viewers show the page rotated by 90°.
\usepackage{pdflscape}

% Advanced tables
\usepackage{array}
\usepackage{tabularx}
\usepackage{longtable}

% Fancy tablerules
\usepackage{booktabs}

% Graphics
\usepackage{graphicx}

% Current time
\usepackage[useregional=numeric]{datetime2}

% Float barriers.
% Automatically add a FloatBarrier to each \section
\usepackage[section]{placeins}

% Custom header and footer
\usepackage{fancyhdr}

\usepackage{geometry}
\usepackage{layout}

% Math tools
\usepackage{mathtools}
% Math symbols
\usepackage{amsmath,amsfonts,amssymb}
\usepackage{amsthm}
% General symbols
\usepackage{stmaryrd}

\DeclarePairedDelimiter\abs{\lvert}{\rvert}

% Indistinguishable operator (three stacked tildes)
\newcommand*{\diffeo}{% 
  \mathrel{\vcenter{\offinterlineskip
  \hbox{$\sim$}\vskip-.35ex\hbox{$\sim$}\vskip-.35ex\hbox{$\sim$}}}}

% Bullet point
\newcommand{\tabitem}{~~\llap{\textbullet}~~}

\pagestyle{plain}
% \fancyhf{}
% \lhead{}
% \lfoot{}
% \rfoot{}
% 
% Source code & highlighting
\usepackage{listings}

% SI units
\usepackage[binary-units=true]{siunitx}
\DeclareSIUnit\cycles{cycles}

% Convenience commands
\newcommand{\mailsubject}{11043 - Internet of Things - Series 10}
\newcommand{\maillink}[1]{\href{mailto:#1?subject=\mailsubject}
                               {#1}}

% Should use this command wherever the print date is mentioned.
\newcommand{\printdate}{\today}

\subject{11043 - Internet of Things}
\title{Series 10}

\author{Michael Senn \maillink{michael.senn@students.unibe.ch} - 16-126-880}

\date{\printdate}

% Needs to be the last command in the preamble, for one reason or
% another. 
\usepackage{hyperref}

\begin{document}
\maketitle


\setcounter{chapter}{9}

\chapter{Series 10}

\section{COAP methods}

The four methods supported by COAP are:
\begin{description}
		\item[GET] Analogous to HTTP GET. Retrieves information stored at a
				given COAP URI.
		\item[POST] Analogous to HTTP POST. Asks for requested COAP URI to
				process data enclosed in request. Can lead to eg a new resource
				being created, or an action being taken.
		\item[PUT] Analogous to HTTP PUT. Asks for resource at requested URI
				to be updated based on the enclosed payload. Usually used to
				change the details of an existing resource.
		\item[DELETE] Analogous to HTTP DELETE. Asks that the resource at the
				requested URI be deleted.
\end{description}

\section{PubSub in MQTT}

The publish-subscriber pattern as used by MQTT works by allowing clients who
publish messages on a broker - `publishers' - to specify which channel to
publish them to.  Clients who want to receive messages from a broker -
`subscribers' - then specify from which channels they want to receive messages.

This has the advantage of completely decoupling publishers and subscribers. A
publisher must only know a suitable channel to publish their message to, and
any subscriber interested in these messages can subscribe and unsubscribe from
them at will.

This also allows implementing fan-out or fan-in communications, where eg a
server pushes commands to a channel listened to by all devices, respectively
devices push updates to a channel listened to by one server.

Authorization can be handled in a centralised way by the broker, based on
whatever mechanism is deemed suitable.

\end{document}
