\documentclass[a4paper]{scrreprt}

% Uncomment to optimize for double-sided printing.
% \KOMAoptions{twoside}

% Set binding correction manually, if known.
% \KOMAoptions{BCOR=2cm}

% Localization options
\usepackage[english]{babel}
\usepackage[T1]{fontenc}
\usepackage[utf8]{inputenc}

% Quotations
\usepackage{dirtytalk}

% Floats
\usepackage{float}

% Enhanced verbatim sections. We're mainly interested in
% \verbatiminput though.
\usepackage{verbatim}

% Automatically remove leading whitespace in lstlisting
\usepackage{lstautogobble}

% PDF-compatible landscape mode.
% Makes PDF viewers show the page rotated by 90°.
\usepackage{pdflscape}

% Advanced tables
\usepackage{array}
\usepackage{tabularx}
\usepackage{longtable}

% Fancy tablerules
\usepackage{booktabs}

% Graphics
\usepackage{graphicx}

% Current time
\usepackage[useregional=numeric]{datetime2}

% Float barriers.
% Automatically add a FloatBarrier to each \section
\usepackage[section]{placeins}

% Custom header and footer
\usepackage{fancyhdr}

\usepackage{geometry}
\usepackage{layout}

% Math tools
\usepackage{mathtools}
% Math symbols
\usepackage{amsmath,amsfonts,amssymb}
\usepackage{amsthm}
% General symbols
\usepackage{stmaryrd}

\DeclarePairedDelimiter\abs{\lvert}{\rvert}

% Indistinguishable operator (three stacked tildes)
\newcommand*{\diffeo}{% 
  \mathrel{\vcenter{\offinterlineskip
  \hbox{$\sim$}\vskip-.35ex\hbox{$\sim$}\vskip-.35ex\hbox{$\sim$}}}}

% Bullet point
\newcommand{\tabitem}{~~\llap{\textbullet}~~}

\pagestyle{plain}
% \fancyhf{}
% \lhead{}
% \lfoot{}
% \rfoot{}
% 
% Source code & highlighting
\usepackage{listings}

% SI units
\usepackage[binary-units=true]{siunitx}
\DeclareSIUnit\cycles{cycles}

% Convenience commands
\newcommand{\mailsubject}{11043 - Internet of Things - Series 7}
\newcommand{\maillink}[1]{\href{mailto:#1?subject=\mailsubject}
                               {#1}}

% Should use this command wherever the print date is mentioned.
\newcommand{\printdate}{\today}

\subject{11043 - Internet of Things}
\title{Series 7}

\author{Michael Senn \maillink{michael.senn@students.unibe.ch} - 16-126-880}

\date{\printdate}

% Needs to be the last command in the preamble, for one reason or
% another. 
\usepackage{hyperref}

\begin{document}
\maketitle


\setcounter{chapter}{6}

\chapter{Series 7}

\section{Sources of energy waste in MAC protocols}

Potential sources of energy waste in MAC protocols are:

\begin{description}
		\item[Collisions and retransmissions] As a physical channel can
				typically only accomodate a finite (often only one)
				simultaneous transmission, collisions cause all senders to have
				to retransmit at a later time. This introduces latency and
				wastes energy.
		\item[Idle listening] Typically receiving and sending data is
				comparably energy-intensive, and significantly more expensive
				than being in an idle state. As such listening for a
				transmission when there is none can waste a lot of energy.
		\item[Overhearing] When listening to a transmission not destined for
				the current node, energy is wasted receiving and processing the
				data. Ideally nodes will not be listening at all when a
				transmission doesn't concern them, and at the very least they
				must be able to exit processing as early as possible.
\end{description}

\section{MaxMAC vs WiseMAC}

\subsection{WiseMAC}

With WiseMAC, all nodes sample the medium with the same frequency but are
initially not synchronized. They learn about others' sampling schedule by means
of their acknowledgment packets. Now posessing the knowledge of when the
recipient will listen, the sender knows when to send the preamble followed by
the data such that the preamble can be as short as possible.

Having received the data teh receiving node then acknowledges having received
data, upon which the sending node can terminate the sending routine.

\subsection{MaxMAC}

MaxMAC builds on WiseMAC, but uses dynamic sampling frequencies. When traffic
on the medium over a sliding window is high, wakeup frequencies are increased.
Similarly when traffic is low, sampling frequencies are lowered.

\section{Adaptability of LEACH vs X-MAC to changes of topology}

\subsection{LEACH}

LEACH is a scheduled protocol where nodes organize themselves in clusters, with
alternating nodes taking on the role of cluster head. Regular nodes get
assigned slots by the cluster head, during which they wake up and will send and
receive data. Any transmission goes to the cluster head which will aggregate
data and forward it to other cluster heads or the sink.

As such changes in topology can cause clusters having to reform, or a
connection to the sink to be interrupted if e.g. a cluster head dies.

\subsection{X-MAC}

X-MAC is a contention-based protocol, where a transmission is initiated by a
preamble containing the destination's address. If the receiving node
acknowledges this preamble, the sending node will transmit data. As such X-MAC
is completely topology-unaware, with the sender communicating directly with the
receiver. Any change in topology will only affect connections where one of the
two nodes was changed.

\end{document}
