\section{Introduction}

\subsection{Sensor networks}

\begin{description}
		\item[Sensor network] Deployment of small, inexpensive self-powered
				devices which can sense, communicate and compute. Size-limited
				by battery.
		\item[Sink] Gateway between fixed and wireless sensor network. Controls
				and manages sensor nodes.
		\item[Challenges] Finite energy \& computation resources, network
				capabilities, unreliable hardware and network, need for
				synchronization (time, location)
		\item[Data dissemination] Observer-initiated (on-demand), event-driven
				(by sensor nodes), continuous (pre-specified rate)
		\item[Transmission and reception costs] Much higher than computations
\end{description}

\subsection{Advanced WSN structures}

\begin{description}
		\item[Mobile WSN] Static: Need dense deployment. Fully mobile
				(sensors are moving): Weight constraints, communication
				challenging. Hybrid (fixed base stations, mobile sensors
				carried by humans/vehicles/...)
		\item[Participatory sensing] Using sensors of end-user devices as part
				of network
		\item[WSAN]: Includes actuators, acting on sensor data
		\item[M2M communications]
		\item[IoT] Physical things having counterpart in network, exposing
				data, acting on commands.
		\item[CPS] Control/computing co-design, actor/sensor networks
		\item[Multimedia WSN] Inclusion of multimedia traffic, e.g.
				surveillance. Requirements for encoding, bandwith, QoS
\end{description}

\subsection{WSN Applications}

WSN applications can be classified by goal, connectivity, mobilitiy, data
rates, ...

\begin{description}
		\item[Querying] Sensors collect data, applications queries data from
				sensors. Potentially aggregation/filtering within WSN.
		\item[Tasking] Sensors programmed to perform actions on events, e.g.
				send message when reading exceeds threshold. Combined with
				actuators (e.g. building control)
\end{description}
